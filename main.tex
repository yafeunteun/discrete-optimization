\documentclass[a4paper]{article} 
\usepackage{listings}
\usepackage{color}

\lstset{frame=tb,
language=R,
keywordstyle=\color{blue},
alsoletter={.}
}


\usepackage{tcolorbox}
\usepackage{array}
\tcbuselibrary{skins}
\usepackage[utf8]{inputenc}
\title{
\vspace{-3em}
\begin{tcolorbox}
\Huge\sffamily Discrete Optimization   % Title of the document
\end{tcolorbox}
\vspace{-3em}
}

\date{}

\usepackage{background}
\SetBgScale{1}
\SetBgAngle{0}
\SetBgColor{red}
\SetBgContents{\rule[0em]{4pt}{\textheight}}
\SetBgHshift{-2.3cm}
\SetBgVshift{0cm}
\usepackage{lipsum}% just to generate filler text for the example
\usepackage[margin=2cm]{geometry}
\usepackage{lipsum}% just to generate dummy text for the example


%\url{http://tex.stackexchange.com/a/314/86}

\makeatletter
\def\cornell{\@ifnextchar[{\@with}{\@without}}
\def\@with[#1]#2#3{
\begin{tcolorbox}[enhanced,colback=gray,colframe=black,fonttitle=\large\bfseries\sffamily,sidebyside=true, nobeforeafter,before=\vfil,after=\vfil,colupper=blue,sidebyside align=top, lefthand width=.3\textwidth,
opacityframe=0,opacityback=.3,opacitybacktitle=1, opacitytext=1,
segmentation style={black!55,solid,opacity=0,line width=3pt},
title=#1
]
\begin{tcolorbox}[colback=red!05,colframe=red!25,sidebyside align=top,
width=\textwidth,nobeforeafter]#2\end{tcolorbox}%
\tcblower
\sffamily
\begin{tcolorbox}[colback=blue!05,colframe=blue!10,width=\textwidth,nobeforeafter]
#3
\end{tcolorbox}
\end{tcolorbox}
}
\def\@without#1#2{
\begin{tcolorbox}[enhanced,colback=white!15,colframe=white,fonttitle=\bfseries,sidebyside=true, nobeforeafter,before=\vfil,after=\vfil,colupper=blue,sidebyside align=top, lefthand width=.3\textwidth,
opacityframe=0,opacityback=0,opacitybacktitle=0, opacitytext=1,
segmentation style={black!55,solid,opacity=0,line width=3pt}
]

\begin{tcolorbox}[colback=red!05,colframe=red!25,sidebyside align=top,
width=\textwidth,nobeforeafter]#1\end{tcolorbox}%
\tcblower
\sffamily
\begin{tcolorbox}[colback=blue!05,colframe=blue!10,width=\textwidth,nobeforeafter]
#2
\end{tcolorbox}
\end{tcolorbox}
}
\makeatother

\parindent=0pt

%\newcommand{\cornell}[2]

%\AddEverypageHook{
%\hspace{.3\textwidth}\vrule width 3pt depth .4\textheight 
%\vspace{-\textheight}}

\providecommand{\LyX}{L\kern-.1667em\lower.25em\hbox{Y}\kern-.125emX\@}

\begin{document} 
\maketitle
\SetBgContents{\rule[0em]{4pt}{\textheight}}


\part{Constraint Programming}


\section{Computational Paradigm}

\cornell{Branch and prune}{
\begin{itemize}
    \item \textbf{pruning}: reduce the search space as much as possible.
    \item \textbf{branching}: decompose the problem into subproblems and explore subproblems
\end{itemize}}

\cornell{Pruning}{
Use constraints to remove, from the variable domains, values that cannot belong to any solution.
}

\cornell{Branching}{
E.g., try all possible values of a variable until a solution is found or it can be proven that no solution exists.
}


\cornell{What does a constraint do?}{
\begin{itemize}
    \item feasibility checking: can a constraint be satisfied given the values in the domains of its variable
    \item pruning: if satisfiable, determine which values in the domains cannot be part of the solution
    \item the algorithms use dedicated algorithms for each constraint
\end{itemize}}


%\cornell{Another research question very long for one line.}{\lipsum*[1]}
%\cornell{Do you want fill much more than a page}{No problem. \lipsum*[3]}
%\cornell{More?}{No problem. \lipsum*[3-4]}

%\cornell[Last question]{This is the end?}{Yes.}
%\cornell{Supervised learning}{}
%\cornell{regression problem}{}
%\cornell{classification problem}{}
\end{document}


}


